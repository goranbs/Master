\documentclass[twoside,english]{uiofysmaster}

%\bibliography{references}

\begin{document}

\section{The Clay Force Field model}

The Clay Force Field model, or ClayFF, were developed by Cygan et.al \textbf{(Reference...)} to effectivly model the behaviour of clay minerals in molecular dynamics simulations. It consists of four potentials, the \textit{van Der Waals} potential, a \textit{Bond} potential, \textit{Coulomb} and a \textit{Angle} potential, along with force-field interaction coefficients for the potentials for a range of atoms.

\begin{equation}
E_{VDW} = \sum _{i\neq j} D_{O,ij}\Big[ \Big( \frac{R_{O,ij}}{r_{ij}} \Big) ^{12} - \Big( \frac{R_{O,ij}}{r_{ij}} \Big) ^{6} \Big]
\label{equation:Cygan_LJ}
\end{equation}

\begin{equation}
E_{coul} = \frac{e^2}{4\pi \epsilon _0}\sum _{i\neq j} \frac{q_i q_j}{r_{ij}}
\label{equation:Cygan_coul}
\end{equation}

\begin{equation}
E_{bond} = k_1(r_{ij}-r_o)^2
\label{equation:Cygan_bond}
\end{equation}

\begin{equation}
E_{angle} = k_2(\theta _{ijk} - \theta _o)^2
\label{equation:Cygan_angle}
\end{equation}

\begin{equation}
R_{O,ij} = \frac{1}{2}(R_{O,i} + R_{O,j})
\label{equation:arithmetic_avg}
\end{equation}

\begin{equation}
D_{O,ij} = \sqrt{D_{O,i}D_{O,j}}
\label{equation:geometric_avg}
\end{equation}

In \lammps, they use the Lennard-Jones potential in \ref{equation:LAMMPS_LJ}, which results in the relationship $ R_{o,ij} = 2^{1/6}\sigma $, and, although the conversionfactor is small, it is extremely important for the simulation. A value of $ R_{o,ij} = 3.5532 $, gives then a LAMMPS input value of $ \sigma = 3.166 $. The relationship factor can easly be calculated by setting both \ref{equation:Cygan_LJ,equation:LAMMPS_LJ} to zero, and solve for the equilibrium distance parameter, you can assume that $ \D_{o,ij} = 4\epsilon $, or don't, cause it falls out either way!

\beq
E_{LJ} = 4\epsilon \Big[ \Big( \frac{\sigma}}{r_{ij}} \Big) ^{12} - 2\Big( \frac{\sigma}{r_{ij}} \Big) ^{6} \Big]
\eeq


%\begin{center}
\begin{table}
 \caption{Force Field Parameters for $CO_2$. (ref Cygan)}
 \begin{tabular}[]{|c|l|}
  \hline
  \multicolumn{2}{|c|}{Nonbond} \\ \hline
  $q_C$ & +0.6512 e  \\ \hline
  $q_O$ & -0.3256 e  \\ \hline
  $\epsilon _C$ & 0.0559 Kcal/mol \\ \hline
  $\epsilon _O$ & 0.1597 Kcal/mol \\ \hline
  $\sigma _C$ & 2.800 \AA{} \\ \hline
  $\sigma _O$ & 2.800 \AA{} \\ \hline
 \end{tabular}
  \begin{tabular}[]{|c|l|}
  \hline
  \multicolumn{2}{|c|}{Bond} \\ \hline
  $k_{CO}$ & 2017.93 Kcal/mol/\AA{}$^2$  \\ \hline
  $r_{o,CO}$ & 1.162 \AA{}  \\ \hline
  $k_{OCO}$ & 108.0 Kcal/mol/rad$^2$ \\ \hline
  $\theta _{0,OCO}$ & 180.0 deg \\ \hline
                    &           \\ \hline
                    &           \\ \hline

 \end{tabular}
 \label{table:ForceFieldParameters_CO2}
\end{table}
%\end{center}

\begin{table}
 \caption{Force Field Parameters for $H_2O$. (ref ?) }
  \begin{tabular}[]{|c|l|}
  \hline
  \multicolumn{2}{|c|}{Nonbond} \\ \hline
  $q_H$ & +0.4238 e  \\ \hline
  $q_O$ & -0.8476 e  \\ \hline
  $\epsilon _H$ & 0.0 Kcal/mol \\ \hline
  $\epsilon _O$ & 0.1553 Kcal/mol \\ \hline
  $\sigma _H$ & 2.058 \AA{} \\ \hline
  $\sigma _O$ & 3.5532 \AA{} \\ \hline
 \end{tabular}
  \begin{tabular}[]{|c|l|}
  \hline
  \multicolumn{2}{|c|}{Bond} \\ \hline
  $k_{OH}$ & 554.1349 Kcal/mol/\AA{}$^2$  \\ \hline
  $r_{o,OH}$ & 1.0 \AA{}  \\ \hline
  $k_{HOH}$ & 45.7696 Kcal/mol/rad$^2$ \\ \hline
  $\theta _{0,HOH}$ & 109.47 deg \\ \hline
                    &           \\ \hline
                    &           \\ \hline

 \end{tabular}
 \label{table:ForceFieldParameters_H2O}
\end{table}

\begin{table}
 \caption{Force Field Parameters for Portlandite $Ca(OH)_2$ (ref Cygan)}
  \begin{tabular}[]{|c|l|}
  \hline
  \multicolumn{2}{|c|}{Nonbond} \\ \hline
  $q_{Ca}$ & +1.050 e  \\ \hline
  $q_O$ & -0.820 e  \\ \hline
  $q_H$ & +0.425 e  \\ \hline

  $\epsilon _{Ca}$ & 5.0298E-6 Kcal/mol \\ \hline
  $\epsilon _H$ &  \\ \hline
  $\epsilon _O$ & 0.1554 Kcal/mol \\ \hline
  $\sigma _{Ca}$ & 6.2428 \AA{} \\ \hline
  $\sigma _H$ &  \\ \hline
  $\sigma _O$ & 3.5532 \AA{} \\ \hline
 \end{tabular}
  \begin{tabular}[]{|c|l|}
  \hline
  \multicolumn{2}{|c|}{Bond} \\ \hline
  $k_{OH}$ & 554.1349 Kcal/mol/\AA{}$^2$  \\ \hline
  $r_{o,OH}$ & 1.0 \AA{}  \\ \hline
  $k_{HOH}$ & 45.7696 Kcal/mol/rad$^2$ \\ \hline
  $\theta _{0,HOH}$ & 109.47 deg \\ \hline
                    &           \\ \hline
                    &           \\ \hline
		    &           \\ \hline
                    &           \\ \hline
                    &           \\ \hline

 \end{tabular}
 \label{table:ForceFieldParameters_Portlandite}
\end{table}


\section{The water model}
The model used for simulation of water in this thesis is the \textit{SPC/E} water model. Where we use the parameters in \ref{table:ForceFieldParameters_H2O} as input for the potentials. We use the Coulumb interaction term between all particles along with the Lennard-Jones potential for the oxygen-oxygen interaction, for the oxygen-hydrogen interaction we use a bond potential described in \ref{equation:Cygan_bond} and a three particle interaction in the angle bend term. The difference between the \textit{SPC/E} and the \textit{SPC} water model is only the point charges of the atoms, the reason for this that they are used to simulate water in some different 

\subsection{The Shake algorithm}
The shake algorithm is used to keep the hydrogen atoms rigid, locked in their equilibrium angular positions.

\section{The carbon dioxide model}
The model used for the simulation of carbon dioxide in this thesis is the \textit{EMP 2} model.

\section{Informative mistakes}

\begin{enumerate}
 \item Make sure that you use the correct minus symbol in the \textit{.lt} (moltemplate) files. I managed to use a long minus character in the Carbon Dioxide unit cell template. LAMMPS skips the character. You will notice this when you run the code when LAMMPS gives you a warning about the system not being charge neutral.
 \item Make sure to triple check that the units you are using is the same as what you have defined LAMMPS to use. Also make sure what output units LAMMPS will give for all outputs.
 \item Use the LAMMPS internet manual. It is actually pritty good!
 \item If you wish to use non-periodic boundary conditions in one direction of your system, make sure to orient your system in the z-direction. A lot of functionalities only support non-periodic boundary conditions this way, like Ewald, and some fixes.
\end{enumerate}


\end{document}
